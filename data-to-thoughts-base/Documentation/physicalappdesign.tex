\section{Physical Application Design}
My thought experiments site has a few main applications that leverage the underlying
database.
The first is the ability to create a new thought experiment:
one specifies a title, category, text and text category,
and optionally media attachments in the form of relative paths or URLs.
Inserting a new thought experiment affects many tables:
\begin{itemize}
    \item ThoughtExperiment;
    \item ThoughtCategory (used to populate the category dropdown);
    \item TE\_Cat (creates a record associating the new thought to a category);
    \item TextSection and TE\_TextSec (store the text associated with the thought);
    \item TextCategory (used to populate the category dropdown);
    \item TS\_Cat (creates a record associating the new text to a category); and
    \item (Optionally) Multimedia and TE\_Mult (for Multimedia attachments).
\end{itemize}
I will note that insertion is a fairly complex task.
There are quite a few intricate relationships that must be maintained here
in order to properly insert the entire thought experiment.

Another set of applications involve browsing the entire set of 
thought experiments, text sections, or multimedia.  These applications
affect their respective tables. My site allows a user to easily switch
between viewing these groupings, and relies on the database to populate 
associated views.

Yet another set of applications enable users to comment on
thought experiments, text sections, or multimedia.
Creating comments primarily involves the Comments table,
but also involves the respective tables containing the entities
being commented on.  These tables allow the user to view
a given entity before and during the commenting process.

Lastly, a final set of applications allow users to
add moods to the major three entities.
Moods are populated in a dropdown from the Moods table,
and successful insertions involve creating entries in the appropriate
association tables (TE\_Mood, TS\_Mood, and Mult\_Mood).

In addition to these major applications,
there are also some minor applications interleaved throughout the site.  
For example, the home page counts the number
of thought experiments to give a quick summary of the site's usage.
Such minor applications improve the user experience of the site.
