\section{User's Guide}
My site is relatively easy to use.
Upon accessing the homepage, you will see a summary of the
number of thought experiments currently being hosted.
Clicking on that number will take you to where you may browse the database's
contents.
At the bottom of the screen, you will see four navigation links that take
you to the main parts of the site.
"Create" will take you to where you may create a thought experiment,
"Browse" will take you to where you may browse the database's contents,
and "About" and "Home" take you to the About and Home pages respectively.

\subsection{Creating a New Thought Experiment}
On the Create page, you will first enter a title for the thought
as well as the general category it fits under.
You may then enter text in the text area, and give that a distinct
category as well.
A button at the bottom allows you to add media attachments to the thought
experiment--simply click the button, then enter a title and a URL or relative
path for the multimedia you wish to attach.
When you are all done, click "Submit."
You will be taken to a summary page which will have a link for you to view the thought you created.

\subsection{Viewing and Browsing}
The Browse page allows you to see the database contents.
You may filter the page using the dropdown in the upper left.
For browsing thought experiments, you can click on the title of the
thought to view that thought in greater detail. When viewing
a thought, you will be able to see its information (category, etc)
as well as any associated text and attachments.

\subsection{Commenting}
You may access commenting from one of two places: the Browse page, or from
viewing a particular thought experiment.
The comment links take you to a separate page where you can enter your
comment text, and then submit that text.
In future work, I would like to make this process more streamlined and
find a salient way to display comments.

\subsection{Moods}
You may apply a mood(s) to a given entity from the same pages as in
accessing commenting. This will take you to a separate page
where you will find a dropdown containing the available moods.
Select an appropriate mood, and click "Moodify."
In future work, I would like to make this process more streamlined and
find a salient way to display moods.
